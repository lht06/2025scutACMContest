\problemWithTimeMem{开盒达人}{2 seconds}{1024MB}

高级加密标准(AES,Advanced Encryption Standard)作为主流对称加密算法,AES以16字节构成的$4\times 4$字节矩阵分组处理数据,执行10-14轮加密。(除最后一轮外)每轮操作包含:

1. \textbf{轮密钥加(AddRoundKey)}:输入与轮密钥异或;

2. \textbf{字节替换(SubBytes)}:通过S盒完成非线性变换,其中的S盒可逆(双射);

3. \textbf{行移位(ShiftRow)}:将$4\times 4$状态矩阵的第i行循环左移i字节;

4. \textbf{列混淆(MixColumn)}:在GF($2^8$)有限域上对每列进行矩阵乘法(加法为异或,乘法按特定多项式规则)。

\textbf{输入数据排布}

$$
\begin{array}{|c|c|c|c|}
\hline
A[0] & A[4] & A[8]  & A[12] \\
\hline
A[1] & A[5] & A[9]  & A[13] \\
\hline
A[2] & A[6] & A[10] & A[14] \\
\hline
A[3] & A[7] & A[11] & A[15] \\
\hline
\end{array}
$$

\textbf{轮密钥加}

\begin{minted}{python}
for i in range(16):
    A[i] ^= Key[i]
\end{minted}

\textbf{字节替换}

\begin{minted}{python}
for i in range(16):
    A[i] = S[A[i]]
\end{minted}

\textbf{行位移变换}

$$
\begin{array}{|c|c|c|c|}
\hline
A[0] & A[4] & A[8]  & A[12] \\
\hline
A[5] & A[9] & A[13]  & A[1] \\
\hline
A[10] & A[14] & A[2] & A[6] \\
\hline
A[15] & A[3] & A[7] & A[11] \\
\hline
\end{array}
$$

\textbf{列混淆矩阵}

$$
\begin{array}{|c|c|c|c|}
\hline
2 & 3 & 1 & 1 \\
\hline
1 & 2 & 3 & 1 \\
\hline
1 & 1 & 2 & 3 \\
\hline
3 & 1 & 1 & 2 \\
\hline
\end{array}
$$

\textbf{$GF(2^8)$有限域乘法实现}

\begin{minted}{python}
def gMul(a, b):
    p = 0
    for i in range(8):
        if b & 1 != 0:
            p ^= a
        hi = a & 0x80
        a <<= 1
        if hi != 0:
            a ^= 0x1b
        b >>= 1
    return p & 255
\end{minted}

\textbf{白盒密码背景}

针对密钥可能暴露在不可信环境(如易被逆向的软件),白盒密码通过混淆轮密钥与算法逻辑,抵御内存窃取攻击,确保密钥在非安全设备中仍受保护。

你作为开盒达人,最擅长破解白盒密码。下面你将得到S盒和若干组AES轮函数的输入和输出,你需要根据每对轮函数的输入和输出,还原出轮密钥(并非最后一轮)。

\mysec{Input}

第一行一个正整数$T\leqslant 1145$,表示有几组输入输出对。

第二行$256$个16进制数字,表示S盒,保证$0\leqslant S[i]\leqslant 255$且不重复。

接下来$T$行,每行$32$个16进制数字,前$16$个为输入的$16$个字节,后$16$个为输出的$16$个字节。

\mysec{Output}

$T$行,每行$16$个$16$进制数,表示该轮的轮密钥。

\ACMIO{Sample 1}{%
1

0x63 0x7c 0x77 0x7b 0xf2 0x6b 0x6f 0xc5(后省略,具体查看oj)
}{%
0x3e 0xec 0xd2 0x2f 0xb4 0x91 0x77 0x2c 0x31 0xe3 0x1a 0xe5 0xba 0x9d 0x44 0x90
}

\mysec{Hint}

构造对应的逆运算。
