\problemWithTimeMem{演绎法}{1 second}{1024MB}

“排除所有不可能的,剩下的那个即使再不可思议,那也是事实。”
 
“夏洛克,你看,我学着你的演绎法,关于哈德森太太的茶杯丢失那件事推理了一下。”

“嗯,华生,你开始学着动动脑筋了?”

福尔摩斯抓过华生递过来的笔记本,选择性地忽视了华生的臭脸。

“推理得十分有趣,思路十分发散,就是有一处矛盾。很不幸,许多东西需要推倒重来了。

“不过找到错误也是一种成功,至少对你来说是这样。相信凭你自己是不够的,我来给你些提示。来吧,找到矛盾吧。”

\includegraphics[width=.8\linewidth]{image5.png}

华生医生推理过程中的思维表现为若干个节点,之间的逻辑推理用连接两个思维的有向边表示。华生医生的思维足够清晰,所以不会有节点推理指向自己,以及节点重复推理直接指向同一个另一节点的情况(但重复间接指向仍有可能发生)。用更加图论的说法,就是无自环无重边。在所有的矛盾点中,\textbf{有两个不同节点相互矛盾},如果某个节点能够经历若干次推理,\textbf{直接或间接地指向这一对矛盾节点},或者\textbf{本身就是矛盾节点的同时指向另一个矛盾节点},那么这个节点就是\textbf{错误的}。福尔摩斯会告诉华生医生所有错误节点,请找出矛盾的两个节点吧。如有多组答案正确,输出任意一组均可。

\mysec{Input}

第一行,两个整数$n,m(2\leqslant n\leqslant 200,1\leqslant m\leqslant \frac{n\times n-n}{2})$,分别表示点的数量和边的数量。

接下来$m$行,每行两个整数$u,v(1\leqslant u\leqslant n,1\leqslant v\leqslant n)$,其中$u$和$v$分别表示这条有向边的起点和终点。

接下来一行,一个整数$k(1\leqslant k\leqslant n)$,表示错误节点的数量。

接下来一行,$k$个整数,代表所有错误节点的序号$a_1,a_2,...,a_k(\forall i\in [1,k],1\leqslant a_i\leqslant n;\forall i\forall j\in [1,k]\land i\neq j,a_i\neq a_j)$。保证所有错误节点序号不重复,且大小介于$1$到$n$之间。

\mysec{Output}

输出一行,两个不同整数,代表两个矛盾节点。

\ACMIO{Sample 1}{%
5 5

1 2

2 3

1 5

4 3

4 5

1

1
}{%
2 5
}

\ACMIO{Sample 2}{%
5 5

1 2

2 3

1 5

4 3

4 5

2

1 4
}{%
3 5
}

\ACMIO{Sample 3}{%
4 3

1 3

2 3

3 4

1

1
}{%
1 3
}

\mysec{Hint}

数据范围:$n,m(2\leqslant n\leqslant 200,1\leqslant m\leqslant \frac{n\times n-n}{2})$。

样例解释:样例1中当矛盾节点为$2$和$5$时,$1$号是错误节点,符合题意。

提示:福尔摩斯不会遗漏错误节点。