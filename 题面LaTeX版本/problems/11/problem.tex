\problemWithTimeMem{又一个计数问题}{6 seconds}{1024MB}

如果一个数字$x(0\leqslant x \leqslant 9)$在数组的每个元素的数位中都出现过,则该数字为这个数组的“主导数字”。例如数组\verb|[3015 1398 301]|有主导数字1和3。
给定$l,r,n,d$,请你统计满足以下条件的严格递增数组$X=[X_1,X_2,...,X_n]$的数量:

1.数组长度为$n$。

2.数组恰好有$d$个主导数字。

3.满足$l\leqslant X_1<X_2<...<X_n\leqslant r$。

由于符合条件的数组数量可能非常多,所以答案对$10^9+7$取模。

\mysec{Input}

第一行包含一个整数 $T$($1 \leqslant T \leqslant 1000$  ),接下来是$T$个测试样例。

对于每个测试样例,输入一行四个整数$l,r,n,d$。($1\leqslant l \leqslant r\leqslant 10^{18},2\leqslant n \leqslant 10,0\leqslant d\leqslant 10$)

\mysec{Output}

对于每个测试样例,输入一行一个数字表示所求答案,答案对$10^9+7$取模。

\ACMIO{Sample 1}{%
5

1 6 2 0

12 31 2 2

9 38 3 2

6 31 2 2

6 23 4 0
}{%
15

2

0

2

2725
}
