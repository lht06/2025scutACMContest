\problemWithTimeMem{Butterfly Delusion}{2 seconds}{1024MB}

在白玉楼的庭院中,Youmu正在向Yuyuko展示她新习得的剑术。练习场地可以看作一个$n\times n$的点阵,最左下角坐标为$(1,1)$,右上角坐标为$(n,n)$,当Youmu初始在场地中的某坐标处时,她可以从与坐标轴平行的四个方向中选择一个冲刺任意长度进行斩击,但不能冲出练习场地外。由于Youmu的冲刺速度达到了光速,她不能在冲刺过程中改变方向;幸运的是,Yuyuko借来了Yakumo的力量,可以在指定的坐标上打开空间裂隙,当Youmu经过空间裂隙时,可以再次调整自己的方向为任意平行于坐标轴的方向,当然也可以保持不变。

初始时,场地上不存在任何空间裂隙。Yuyuko会发出$m$条指令,指令包括两种:第一种指令会指定坐标$(x,y)$,在$(x,y)$处打开空间裂隙,注意操作对于场地的影响是永久的;第二种指令会为Youmu指定初始坐标$(x_1,y_1)$和目标坐标$(x_2,y_2)$。对于所有第二种指令,你需要判断Youmu能否通过一次冲刺到达目的地。

\mysec{Input}

本题每个测试样例包括多组数据;

每个测试用例第一行为一个整数$T(1\leqslant T\leqslant 10^5)$,表示测试数据组数;

每组数据第一行包括两个整数$n(1\leqslant n \leqslant 10^6)$,$m(1\leqslant m \leqslant 10^6)$,分别表示场地大小和指令条数;

接下来$m$行,每行描述一条指令,第一个整数$opt(1\leqslant opt \leqslant 2)$表示指令类型,若$opt=1$,表示第一种指令,接下来输入两个整数$x(1\leqslant x \leqslant n),y(1\leqslant y \leqslant n)$,表示在坐标$(x,y)$处打开一条空间裂隙,输入数据保证不会在同一坐标处打开两次空间裂隙;若$opt=2$,表示第二种指令,接下来输入四个整数$x_1(1\leqslant x_1 \leqslant n),y_1(1\leqslant y_1 \leqslant n),x_2(1\leqslant x_2 \leqslant n),y_2(1\leqslant y_2 \leqslant n)$,表示起点坐标为$(x_1,y_1)$,终点坐标为$(x_2,y_2)$。

保证每个测试用例中$\sum{n}\leqslant 10^6,\sum{q}\leqslant 10^6$。

\mysec{Output}

对于每个第二种指令,输出一行字符串,若Youmu可以通过一次冲刺达到目的地,输出"YES",否则输出"NO"。注意字母均为大写。

\ACMIO{Sample 1}{%
1

100 7

2 1 2 3 2

2 2 5 5 4

1 5 5

2 2 5 5 4

2 2 5 10 4

1 5 4

2 2 5 10 4
}{%
YES

NO

YES

NO

YES
}

\ACMIO{Sample 2}{%
1

1 3

2 1 1 1 1 

1 1 1

2 1 1 1 1
}{%
YES

YES
}

\mysec{Hint}

对于第一个样例的第二个询问,由于从$(2,5)$出发无法在不改变方向的情况下到达$(5,4)$,因此答案为NO;而在$(5,5)$处打开隙间后,可以通过路径$(2,5)->(5,5)->(5,4)$到达$(5,4)$,因此答案为YES;

对于最后一个询问,可以通过路径$(2,5)->(5,5)->(5,4)->(10,4)$到达终点,因此答案为YES

本题输入/输出数据量较大,建议使用效率较高的输入/输出方式